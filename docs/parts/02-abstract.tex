\section*{\centering РЕФЕРАТ}
\addcontentsline{toc}{section}{РЕФЕРАТ}
\setcounter{page}{2}

Расчетно-пояснительная записка \pageref{LastPage} с., \totalfigures\ рис., 4 лист., 2 ист.

РАСПИСАНИЕ, СТАНЦИЯ, .NET, С\#, WINFORMS

Целью данной работы является разработка программы для составления расписания движения по железнодорожной станции.

В процессе работы былы определены стуктуры данных, описывающие объекты в станции.
Был введен алгоритм для управления движением.
Были введены критерия для распределения поездов.
Были выбраны технологии для решения поставленной задачи.
Для визуализации работы алгоритма была разработана соответствующая программа.

%Проведено исследование быстродействия функции на стороне базы данных при большом обьеме данных.  
%Из результатов исследования следует, что время выполнения функции увеличивается при увеличении объема данных.