\section*{\centering ВВЕДЕНИЕ}
\addcontentsline{toc}{section}{ВВЕДЕНИЕ}

Железнодорожный транспорт является важнейшим звеном в системе грузовых и пассажирских перевозок.
Эффективное управление движением поездов на станциях является важной задачей, особенно в условиях высокой загруженности и необходимости избегать задержек.
Правильное распределение поездов по доступным платформам позволяет минимизировать время ожидания и исключить ситуации, когда поезда вынуждены останавливаться перед станцией в ожидании освобождения платформы.

Целью данной работы является разработка программы для составления расписания движения по железнодорожной станции.

Для достижения поставленной цели необходимо решить следующие задачи:
\begin{itemize}
	\item описать структуры данных, храняющие информации о поезде, пути, платформе, станции;
	\item определить критерии, по которым будет осуществляться выбор платформы для поезда;
	\item разработать алгоритм, который будет автоматически распределять поезда по платформам в зависимости от их направления;
	\item создать программное обеспечение, обеспечивающее демонстрации работы алгоритма;
\end{itemize}